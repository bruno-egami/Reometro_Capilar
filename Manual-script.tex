\documentclass[a4paper,12pt]{article}
\usepackage{amsmath,amsfonts,siunitx}
\usepackage{geometry}
\usepackage{graphicx}
\usepackage{booktabs}
\usepackage{hyperref}
\usepackage{enumitem}
%\usepackage{titlesec}
\usepackage[utf8]{inputenc}
\usepackage[T1]{fontenc}
\usepackage{indentfirst}
\geometry{margin=2.5cm}
\sisetup{locale = FR, per-mode=symbol, separate-uncertainty = true}
%\titleformat{\section}{\normalfont\Large\bfseries}{\thesection}{1em}{}

\title{Análise Reológica de Pastas Cerâmicas}
\author{Bruno Kenji Nishitani Egami}
\date{}
\usepackage{helvet}
\renewcommand{\familydefault}{\sfdefault}

\begin{document}
\maketitle
\href{https://github.com/bruno-egami/Reometro_Capilar-HX711-4xSG350}{\textbf{Link para o projeto}}

\section*{1. Introdução}
A caracterização reológica de pastas cerâmicas é essencial para entender seu comportamento durante processos de conformação, como a extrusão. O presente relatório descreve as equações utilizadas em um script Python desenvolvido para processar os dados experimentais obtidos com um reômetro capilar DIY de baixo custo. O objetivo é converter medidas diretas (como massa extrudada e pressão de extrusão) em parâmetros reológicos interpretáveis (como viscosidade e taxa de cisalhamento).

\subsection*{1.1 Aplicações em Manufatura Aditiva}
O controle reológico preciso é fundamental para tecnologias emergentes como a impressão 3D cerâmica, especialmente na técnica \textit{Direct Ink Writing} (DIW) ou \textit{Robocasting}. Esta técnica requer pastas com comportamento pseudoplástico que fluam sob cisalhamento (extrusão pelo bico), mas recuperem rapidamente sua viscosidade após a deposição para suportar as camadas subsequentes (\textit{buildability}). A caracterização sistemática do comportamento reológico da matéria-prima base é, portanto, um pré-requisito para o desenvolvimento de formulações otimizadas para manufatura aditiva. 

\subsection*{1.2 Histórico de Desenvolvimento do Equipamento}
O reômetro capilar utilizado neste estudo é resultado de um processo evolutivo de três gerações:

\begin{itemize}
  \item \textbf{V1 (Protótipo Inicial):} A primeira versão utilizava medição indireta de pressão através de uma célula de carga (extensômetros e módulo HX711) acoplada ao pistão. Esta configuração permitia estimar a força aplicada durante a extrusão. O sistema foi validado experimentalmente demonstrando boa concordância com reômetros comerciais.
  
  \item \textbf{V2 (Transdutor Único):} Na segunda geração, a célula de carga foi substituída por um transdutor de pressão de linha (faixa de 0-150 psi) para leitura direta da pressão do ar comprimido. Esta modificação simplificou o sistema e melhorou a precisão das medições.
  
  \item \textbf{V3 (Atual - Duplo Transdutor):} A versão atual incorpora um segundo transdutor posicionado diretamente na câmara da pasta (sensor "Pasta"), além do transdutor de linha (sensor "Linha"). Esta configuração permite eliminar erros sistemáticos causados pelo atrito do pistão e possibilita a aplicação mais precisa das correções de Bagley e Mooney, fundamentais para fluidos não-Newtonianos.
\end{itemize} 

\subsection*{1.3 Validação Experimental}
As versões anteriores do reômetro capilar foram validadas experimentalmente por meio de comparação direta com um reômetro comercial de referência (\textbf{Anton Paar MCR 102}), equipado com geometria de placas paralelas. Os ensaios comparativos, realizados com suspensões de caulim a 60\% de sólidos, demonstraram boa concordância entre os equipamentos, especialmente para taxas de cisalhamento elevadas (acima de \SI{100}{s^{-1}}), confirmando a confiabilidade do dispositivo de baixo custo para análises reológicas.

\subsection*{1.4 Dicas Práticas de Preparação de Amostras}
Com base na experiência acumulada durante o desenvolvimento do equipamento, recomenda-se:

\begin{itemize}
  \item \textbf{Homogeneização:} Utilizar sacos plásticos resistentes para misturar caulim e água por agitação manual vigorosa. Este método facilita a homogeneização e permite transferência direta da pasta para o reômetro (cortando uma das extremidades do saco).
  
  \item \textbf{Limites de Concentração:} Para suspensões puras de caulim-água (sem aditivos), o limite prático de concentração é de aproximadamente 65\% de sólidos. Acima deste valor, a pasta torna-se excessivamente rígida e quebradiça ("aspecto de farofa"), inviabilizando a extrusão sem o uso de modificadores reológicos (plastificantes ou defloculantes).
\end{itemize} 

\section*{2. Variáveis e Nomenclatura Utilizada}
\begin{tabular}{@{}ll@{}}
$R$ & Raio interno do capilar (\si{m}) \\
$D$ & Diâmetro interno do capilar (\si{mm}) \\
$L$ & Comprimento do capilar (\si{m}) \\
$Q$ & Vazão volumétrica (\si{m^{3}/s}) \\
$\rho$ & Densidade da pasta (\si{kg/m^{3}}) \\
$m$ & Massa extrudada (\si{kg}) \\
$t$ & Tempo de extrusão (\si{s}) \\
$P$ & Pressão aplicada (\si{Pa}) \\
$\tau_w$ & Tensão de cisalhamento na parede (\si{Pa}) \\
$\dot{\gamma}_{aw}$ & Taxa de cisalhamento aparente (\si{s^{-1}}) \\
$\dot{\gamma}_w$ & Taxa de cisalhamento real (corrigida) (\si{s^{-1}}) \\
$\eta_a$ & Viscosidade aparente (\si{Pa.s}) \\
$\eta$ & Viscosidade verdadeira (\si{Pa.s}) \\
$n'$ & Índice de pseudoplasticidade estimado (adimensional) \\
$K'$ & Coeficiente de consistência estimado (\si{Pa.s^n}) \\
$\tau_0$ & Tensão de escoamento (\si{Pa}) \\
$\eta_p$ & Viscosidade plástica (\si{Pa.s}) \\
\end{tabular}

\section*{3. Conversão de Medidas Experimentais}

Durante os ensaios reológicos capilares, coletam-se dados experimentais em unidades técnicas convencionais — por exemplo, massa em gramas, pressão em bar, e dimensões do capilar em milímetros. No entanto, todas as equações da mecânica dos fluidos que regem a análise exigem que as grandezas estejam expressas em \textbf{unidades do Sistema Internacional (SI)}. Por isso, o primeiro passo do script é converter essas unidades de forma padronizada.

\subsection*{3.1 Conversões de Unidades para SI}

A primeira etapa do processamento consiste em converter todas as grandezas fornecidas pelo usuário para o Sistema Internacional de Unidades (SI). Essa padronização é essencial para garantir a consistência matemática de todas as equações físicas subsequentes.

\textbf{Conversões realizadas:}

\begin{itemize}[leftmargin=1.5em]
  \item \textbf{Comprimento:}
  \begin{equation}
  D~[\si{mm}] \rightarrow D~[\si{m}] \quad \text{via} \quad D = \frac{D_{\text{mm}}}{1000}
  \end{equation}
  \begin{equation}
  L~[\si{mm}] \rightarrow L~[\si{m}] \quad \text{via} \quad L = \frac{L_{\text{mm}}}{1000}
  \end{equation}

  \item \textbf{Massa:}
  \begin{equation}
  m~[\si{g}] \rightarrow m~[\si{kg}] \quad \text{via} \quad m = \frac{m_{\text{g}}}{1000}
  \end{equation}

  \item \textbf{Densidade:}
  \begin{equation}
  \rho~[\si{g/cm^3}] \rightarrow \rho~[\si{kg/m^3}] \quad \text{via} \quad \rho = \rho_{\text{g/cm}^3} \times 1000
  \end{equation}

  \item \textbf{Pressão:}
  \begin{equation}
  P~[\si{bar}] \rightarrow P~[\si{Pa}] \quad \text{via} \quad P = P_{\text{bar}} \times 10^5
  \end{equation}
\end{itemize}

\textbf{Uso nas demais seções:} Todas as equações da Seção 3 até a Seção 8 pressupõem que as variáveis estejam em SI. A partir daqui, assumiremos que todas as grandezas estão corretamente convertidas.

\vspace{1em}

\subsection*{3.2 Volume Extrudado}
\begin{equation}
V = \frac{m}{\rho} \quad \left[ \si{m^{3}} \right]
\end{equation}

\textbf{Onde:}
\begin{itemize}[leftmargin=1.5em]
  \item $V$ – Volume extrudado (\si{m^3})
  \item $m$ – Massa da pasta extrudada (\si{kg})
  \item $\rho$ – Densidade da pasta (\si{kg/m^3})
\end{itemize}

O volume é necessário para calcular a \textbf{vazão volumétrica} na próxima etapa. Isso estabelece o primeiro elo entre a medição direta ($m$) e o comportamento do escoamento.

\vspace{1em}

\subsection*{3.3 Vazão Volumétrica}
\begin{equation}
Q = \frac{V}{t} \quad \left[ \si{m^{3}/s} \right]
\end{equation}

\textbf{Onde:}
\begin{itemize}[leftmargin=1.5em]
  \item $Q$ – Vazão volumétrica da pasta (\si{m^3/s})
  \item $V$ – Volume extrudado (\si{m^3})
  \item $t$ – Tempo de extrusão (\si{s})
\end{itemize}

A vazão ($Q$) representa o volume de pasta que passa por segundo através do capilar. É uma das grandezas mais importantes da análise, pois se relaciona diretamente à deformação do fluido. Essa informação será usada no cálculo da taxa de cisalhamento (Seção 3.4) e, posteriormente, corrigida por efeitos geométricos (Seções 4 a 6).

\vspace{1em}

\subsection*{3.4 Taxa de Cisalhamento Aparente}
\begin{equation}
\dot{\gamma}_{aw} = \frac{4Q}{\pi R^3} \quad \left[ \si{s^{-1}} \right]
\end{equation}

\textbf{Onde:}
\begin{itemize}[leftmargin=1.5em]
  \item $\dot{\gamma}_{aw}$ – Taxa de cisalhamento aparente na parede (\si{s^{-1}})
  \item $Q$ – Vazão volumétrica (\si{m^3/s})
  \item $R$ – Raio interno do capilar (\si{m})
  \item $\pi$ – Constante matemática (\textasciitilde 3{,}1416)
\end{itemize}

Estima a rapidez com que as camadas de fluido deslizam entre si na parede do capilar. É chamada "aparente" porque assume escoamento newtoniano ideal, e será corrigida mais adiante. Essa grandeza é usada como base para:

\begin{itemize}[leftmargin=1.5em]
  \item Aplicar a correção de Weissenberg–Rabinowitsch (Seção 6);
  \item Calcular a viscosidade aparente (Seção 7.1);
  \item Determinar os parâmetros dos modelos reológicos (Seção 8).
\end{itemize}

\vspace{1em}

\subsection*{3.5 Tensão de Cisalhamento na Parede}
\begin{equation}
\tau_w = \frac{P R}{2 L} \quad \left[ \si{Pa} \right]
\end{equation}

\textbf{Onde:}
\begin{itemize}[leftmargin=1.5em]
  \item $\tau_w$ – Tensão de cisalhamento na parede (\si{Pa})
  \item $P$ – Pressão aplicada (\si{Pa})
  \item $R$ – Raio do capilar (\si{m})
  \item $L$ – Comprimento do capilar (\si{m})
\end{itemize}

Quantifica o esforço por unidade de área exercido pela pasta na parede interna do capilar. Esse valor será:

\begin{itemize}[leftmargin=1.5em]
  \item Corrigido pela técnica de Bagley se necessário (Seção 4);
  \item Utilizado diretamente na correção W-R (Seção 6);
  \item Base para o cálculo de viscosidade aparente e verdadeira (Seção 7);
  \item Variável dependente nos ajustes de modelos (Seção 8).
\end{itemize}


\textbf{Nota:} Esse valor é chamado de “bruto” porque pode incluir contribuições não desejadas, como perdas por entrada/saída e deslizamento — que serão corrigidas em seções posteriores.


\section*{4. Correção de Bagley (Perda nas Extremidades)}

A medição da pressão total durante a extrusão capilar ($P$) inclui não apenas o esforço necessário para o escoamento dentro do capilar, mas também perdas associadas às extremidades — isto é, à entrada e à saída do capilar. Essas perdas ocorrem por efeitos geométricos, contrações abruptas, turbulência localizada e atrito de entrada.

A não correção dessas perdas resulta na superestimação da tensão de cisalhamento calculada, especialmente em capilares curtos. Isso compromete os valores de viscosidade e o ajuste dos modelos reológicos.

\subsection*{4.1 Modelo Linear da Correção de Bagley}
\begin{equation}
P = 2 \tau_w \left( \frac{L}{R} \right) + P_e \quad \left[ \si{Pa} \right]
\end{equation}

\textbf{Onde:}
\begin{itemize}[leftmargin=1.5em]
  \item $P$ – Pressão total medida experimentalmente (\si{Pa})
  \item $\tau_w$ – Tensão de cisalhamento na parede (\si{Pa})
  \item $L$ – Comprimento do capilar (\si{m})
  \item $R$ – Raio interno do capilar (\si{m})
  \item $P_e$ – Perda de pressão nas extremidades (\si{Pa})
\end{itemize}

\textbf{Interpretação:} A equação mostra que a pressão total é composta por:
\begin{enumerate}[label=(\alph*)]
  \item Um termo linear proporcional a $L/R$, que representa o escoamento interno.
  \item Um termo constante $P_e$, que representa as perdas localizadas nas extremidades.
\end{enumerate}

\subsection*{4.2 Aplicação Experimental e Limitações}

Para aplicar a correção de Bagley, utiliza-se um conjunto de capilares com:
\begin{itemize}
  \item Mesmo diâmetro interno ($D$);
  \item Comprimentos distintos ($L_i$);
  \item Vários pontos experimentais com massa e pressão para cada capilar.
\end{itemize}

O script executa os seguintes passos:
\begin{enumerate}[leftmargin=2em]
  \item Calcula $\dot{\gamma}_{aw}$ para cada ponto experimental (com $Q$ e $R$ fixos);
  \item Define um conjunto comum de taxas de cisalhamento alvo $\dot{\gamma}_{aw,k}$;
  \item Interpola os dados de pressão para essas taxas alvo em cada capilar;
  \item Ajusta, para cada $\dot{\gamma}_{aw,k}$, uma reta da forma:
  \begin{equation}
  P_k = 2 \tau_{w,k} \left( \frac{L}{R} \right) + P_{e,k}
  \end{equation}
  \item Calcula a inclinação da reta ($\tau_{w,k}$) e o intercepto ($P_{e,k}$);
  \item Constrói a curva de fluxo corrigida: pares $\left( \dot{\gamma}_{aw,k}, \tau_{w,k} \right)$.
\end{enumerate}

Ao remover as perdas de extremidade, a curva de fluxo se torna mais precisa, especialmente em altas pressões e com capilares curtos. Isso garante que os valores de viscosidade e os modelos reológicos reflitam o verdadeiro comportamento do fluido.

Contudo, a correção exige pelo menos dois capilares com diferentes comprimentos e mesmo diâmetro. Se os dados forem escassos ou os comprimentos forem muito próximos, a regressão pode ser imprecisa.



\section*{5. Correção de Mooney (Deslizamento na Parede)}

Em certos materiais — especialmente pastas cerâmicas, argilas e suspensões concentradas — pode ocorrer um fenômeno chamado \textbf{deslizamento na parede} (wall slip). Isso acontece quando a interface entre o fluido e o capilar não apresenta atrito total, fazendo com que o fluido escorregue parcialmente em vez de aderir à parede. Esse deslizamento afeta diretamente a medição da taxa de cisalhamento, levando à superestimação da deformação real do fluido. Em razão deste efeito, a taxa de cisalhamento aparente medida inclui uma componente falsa de movimento, resultando em uma viscosidade artificialmente baixa e modelos que não representam o comportamento real da pasta.

A correção de Mooney é aplicada para avaliar e quantificar esse deslizamento.

\subsection*{5.1 Modelo de Mooney}
\begin{equation}
\label{eq:mooney_model}
\dot{\gamma}_{aw} = \dot{\gamma}_{s,f} + \frac{C}{R} \quad \left[ \si{s^{-1}} \right]
\end{equation}

\textbf{Onde:}
\begin{itemize}[leftmargin=1.5em]
  \item $\dot{\gamma}_{aw}$ – Taxa de cisalhamento aparente na parede (\si{s^{-1}})
  \item $\dot{\gamma}_{s,f}$ – Taxa de cisalhamento verdadeira do fluido (sem efeito de deslizamento) (\si{s^{-1}})
  \item $C$ – Coeficiente de deslizamento (velocidade de deslizamento na parede, $v_s$) (\si{m/s})
  \item $R$ – Raio do capilar (\si{m})
\end{itemize}

Quanto menor o raio $R$, maior será o impacto do termo de deslizamento ($C/R$) na taxa de cisalhamento aparente medida.

\subsection*{5.2 Aplicação Experimental}
O método exige:
\begin{itemize}
  \item Múltiplos capilares com mesmo comprimento $L$ e diferentes diâmetros ($R_i$);
  \item Medidas de $\dot{\gamma}_{aw}$ para uma tensão de cisalhamento na parede ($\tau_w$) constante (ou um conjunto de $\tau_w$ alvo);
  \item Ajuste linear de $\dot{\gamma}_{aw}$ como função de $1/R$.
\end{itemize}

A correção de Mooney é realizada com base em diferentes capilares de mesmo comprimento e raios distintos. Para cada valor de tensão de cisalhamento \( \tau_w \), interpola-se a taxa de cisalhamento aparente \( \dot{\gamma}_{aw} \) em função de \( 1/R \), e ajusta-se uma regressão linear da forma:

\begin{equation}
\dot{\gamma}_{aw} = \dot{\gamma}_s + \frac{2V_s}{R}
\end{equation}

O termo independente da regressão fornece uma estimativa da taxa de cisalhamento devido ao deslizamento na parede \( \dot{\gamma}_s \), indicando a influência de efeitos de parede no escoamento.


Plotando $\dot{\gamma}_{aw}$ versus $1/R$ para um $\tau_w$ fixo, a equação \eqref{eq:mooney_model} representa uma reta. O \textbf{intercepto} dessa reta fornece $\dot{\gamma}_{s,f}$ (a taxa de cisalhamento do fluido corrigida para o deslizamento), e a \textbf{inclinação} fornece $C$ (o coeficiente de deslizamento, ou velocidade de deslizamento na parede). Com isso, pode-se reconstruir a curva de cisalhamento ($\tau_w$ vs $\dot{\gamma}_{s,f}$) sem o efeito de parede.


\subsection*{5.3 Implementação da Correção de Mooney no Script}

A correção de Mooney tem como objetivo estimar o deslizamento do fluido nas paredes do capilar, o que afeta a determinação da taxa de cisalhamento real. Este fenômeno é particularmente relevante em sistemas onde a tensão de cisalhamento na parede ($\tau_w$) não resulta exclusivamente da deformação do fluido, mas também da contribuição do escoamento por deslizamento.

No presente script, a correção é ativada caso o usuário forneça dados de ao menos dois capilares com diâmetros distintos ($D_i$, portanto raios $R_i$ distintos) e comprimento comum ($L$). A metodologia consiste em:

\begin{enumerate}
    \item Para cada capilar, obter pares de dados ($\tau_w, \dot{\gamma}_{aw}$). A tensão $\tau_w$ pode ser proveniente dos dados corrigidos por Bagley (Seção 4), caso essa correção também tenha sido ativada; caso contrário, os valores diretos de $\tau_w$ são utilizados.
    \item Definir um conjunto de valores alvo para a tensão de cisalhamento na parede ($\tau_{w,k}$).
    \item Para cada $\tau_{w,k}$ alvo:
    \begin{itemize}
        \item Interpolar o valor de $\dot{\gamma}_{aw}$ correspondente para cada capilar $i$ (com raio $R_i$).
        \item Se houver valores de $\dot{\gamma}_{aw}$ para pelo menos dois raios $R_i$ distintos, realizar um ajuste linear da forma:
            \begin{equation} \dot{\gamma}_{aw}(R_i) = \dot{\gamma}_{s,f,k} + C_k \left( \frac{1}{R_i} \right) \end{equation}
        \item O \textbf{intercepto} da regressão fornece $\dot{\gamma}_{s,f,k}$ (a taxa de cisalhamento do fluido para o $\tau_{w,k}$ considerado).
        \item A \textbf{inclinação} da regressão fornece $C_k$ (o coeficiente de deslizamento para o $\tau_{w,k}$ considerado).
    \end{itemize}
    \item São aceitos apenas os pontos onde a regressão é bem definida (ex: intercepto fisicamente coerente, $R^2$ satisfatório).
    \item Ao final, a lista de pares corrigidos ($\tau_{w,k}, \dot{\gamma}_{s,f,k}$) é salva para análise e visualização.
\end{enumerate}

Quando a regressão não é possível (por exemplo, devido a dados insuficientes ou inconsistentes para um determinado $\tau_{w,k}$), esse ponto específico da correção de Mooney é descartado, e o processamento para os demais pontos ou correções prossegue com os dados disponíveis.



\section*{6. Correção de Weissenberg–Rabinowitsch}

A equação da taxa de cisalhamento aparente (Seção 3.4) assume que o fluido apresenta um perfil de velocidade parabólico, típico de fluidos newtonianos. No entanto, pastas cerâmicas e materiais reológicos complexos exibem perfis não parabólicos devido à sua viscosidade dependente da taxa de deformação. Isso significa que $\dot{\gamma}_{aw}$ (ou $\dot{\gamma}_{s,f}$ após a correção de Mooney) \textbf{não representa corretamente} a taxa real de cisalhamento na parede do capilar para fluidos não newtonianos.

A correção de Weissenberg–Rabinowitsch (W-R) ajusta esse erro, fornecendo uma estimativa mais realista da taxa de cisalhamento na parede, $\dot{\gamma}_w$, baseada no comportamento não newtoniano do material.
Para a aplicação da correção de Rabinowitsch e obtenção da taxa de cisalhamento verdadeira na parede \( \dot{\gamma}_w \), é necessário estimar o índice de comportamento aparente \( n' \), obtido por regressão linear da relação logarítmica entre tensão e taxa de cisalhamento:

\begin{equation}
\log \tau_w = \log K' + n' \log \dot{\gamma}_{aw}
\end{equation}

O valor de \( n' \) é então utilizado na expressão:

\begin{equation}
\dot{\gamma}_w = \frac{3n' + 1}{4n'} \dot{\gamma}_{aw}
\end{equation}

Essa abordagem é recomendada quando não se conhece a equação constitutiva do fluido e permite uma correção geral da taxa de deformação.


\subsection*{6.1 Fundamento Teórico}

Para fluidos não newtonianos, o perfil de velocidade no capilar não é simétrico ou parabólico. Ele se achata (pseudoplásticos) ou se afunila (dilatantes), alterando a distribuição da velocidade radial e, consequentemente, a taxa de cisalhamento na parede.

A correção W-R introduz um fator de ajuste baseado no \textbf{índice de comportamento de fluxo local}, $n'$, calculado a partir da relação entre tensão de cisalhamento e a taxa de cisalhamento efetiva antes desta correção (que pode ser $\dot{\gamma}_{aw}$ ou $\dot{\gamma}_{s,f}$) em escala logarítmica:

\begin{equation}
n' = \frac{d \ln \tau_w}{d \ln \dot{\gamma}_{\text{antes de W-R}}}
\end{equation}

\textbf{Interpretação:}
\begin{itemize}[leftmargin=1.5em]
  \item Se $n' = 1$, o fluido é newtoniano (sem correção significativa);
  \item Se $n' < 1$, o fluido é pseudoplástico (tende a se afinar com o cisalhamento);
  \item Se $n' > 1$, o fluido é dilatante (tende a engrossar com o cisalhamento).
\end{itemize}

\subsection*{6.2 Equação de Correção da Taxa de Cisalhamento}
\begin{equation}
\dot{\gamma}_w = \frac{3n' + 1}{4n'} \cdot \dot{\gamma}_{\text{antes de W-R}} \quad \left[ \si{s^{-1}} \right]
\end{equation}

\textbf{Onde:}
\begin{itemize}[leftmargin=1.5em]
  \item $\dot{\gamma}_w$ – Taxa de cisalhamento real na parede (\si{s^{-1}})
  \item $\dot{\gamma}_{\text{antes de W-R}}$ – Taxa de cisalhamento aparente ou corrigida por Mooney (\si{s^{-1}})
  \item $n'$ – Índice de comportamento de fluxo local (adimensional)
\end{itemize}

\textbf{Finalidade:} Corrigir a deformação medida levando em conta a forma real do perfil de velocidade no escoamento capilar. Essa taxa corrigida é usada diretamente para:
\begin{itemize}[leftmargin=1.5em]
  \item Calcular a \textbf{viscosidade verdadeira} (Seção 7.2);
  \item Ajustar os \textbf{modelos reológicos} (Seção 8);
  \item Analisar se o fluido tem comportamento pseudoplástico, dilatante ou aproximado de newtoniano.
\end{itemize}

\subsection*{6.3 Como o Script Aplica a Correção}

O script realiza a correção W-R da seguinte maneira:

\begin{enumerate}[leftmargin=2em]
  \item Recebe os pares $(\tau_w, \dot{\gamma}_{\text{base}})$, onde $\dot{\gamma}_{\text{base}}$ são os dados de taxa de cisalhamento após correções anteriores (Bagley e/ou Mooney, se aplicadas) ou os dados brutos de $\dot{\gamma}_{aw}$;
  \item Aplica interpolação ou suavização log-log (dependendo do número de pontos disponíveis);
  \item Calcula $n'$ como a inclinação da curva $\ln(\tau_w)$ versus $\ln(\dot{\gamma}_{\text{base}})$;
  \item Aplica a fórmula de correção para obter $\dot{\gamma}_w$;
  \item Armazena os pares $(\tau_w, \dot{\gamma}_w)$ para uso em etapas posteriores.
\end{enumerate}

\textbf{Observações:}
\begin{itemize}[leftmargin=1.5em]
  \item O cálculo de $n'$ pode ser feito ponto a ponto (derivada numérica) ou globalmente (ajuste linear em escala log-log);
  \item Se houver muitos pontos experimentais, o script opta por um modelo suavizado para evitar instabilidades;
  \item Para garantir precisão, o script evita regiões com taxas de cisalhamento muito baixas (onde erros relativos são maiores).
\end{itemize}

\subsection*{6.4 Impacto Prático da Correção}

\textbf{Sem correção:} A viscosidade seria calculada com base em uma taxa de cisalhamento não representativa do comportamento não newtoniano, levando a:
\begin{itemize}[leftmargin=1.5em]
  \item Subestimação da viscosidade em fluidos dilatantes;
  \item Superestimação em fluidos pseudoplásticos;
  \item Parâmetros incorretos nos modelos ajustados.
\end{itemize}

\textbf{Com correção:} O script fornece uma descrição realista do comportamento do fluido, refletindo a física do escoamento em um tubo capilar, e possibilita a aplicação rigorosa de modelos não newtonianos.

\textbf{Resultado:} Os dados $(\tau_w, \dot{\gamma}_w)$ corrigidos são agora representações confiáveis do comportamento reológico do material e base para as próximas análises quantitativas.


\section*{7. Viscosidades}

A viscosidade é uma das propriedades mais fundamentais na reologia: ela quantifica a resistência interna de um fluido ao escoamento. No entanto, para fluidos não newtonianos, essa resistência depende da taxa de cisalhamento. Por isso, distinguimos dois conceitos:

\begin{itemize}[leftmargin=1.5em]
  \item \textbf{Viscosidade Aparente} – Calculada com base na taxa de cisalhamento aparente ($\dot{\gamma}_{aw}$) e na tensão correspondente ($\tau_w$) sem considerar o perfil real do escoamento ou outras correções.
  \item \textbf{Viscosidade Verdadeira} – Calculada após todas as correções aplicáveis (Bagley, Mooney, W-R), usando a taxa de cisalhamento real na parede ($\dot{\gamma}_w$) e a tensão de cisalhamento na parede corrigida ($\tau_w$), representando melhor o comportamento físico real do material.
\end{itemize}

\subsection*{7.1 Viscosidade Aparente}
\begin{equation}
\eta_a = \frac{\tau_w}{\dot{\gamma}_{aw}} \quad \left[ \si{Pa.s} \right]
\end{equation}

\textbf{Onde:}
\begin{itemize}[leftmargin=1.5em]
  \item $\eta_a$ – Viscosidade aparente (\si{Pa.s})
  \item $\tau_w$ – Tensão de cisalhamento na parede (não corrigida por Bagley) (\si{Pa})
  \item $\dot{\gamma}_{aw}$ – Taxa de cisalhamento aparente (não corrigida por Mooney ou W-R) (\si{s^{-1}})
\end{itemize}

Este valor representa a “resistência ao escoamento” calculada diretamente dos dados brutos (após conversão de unidades). Ela é útil como uma estimativa inicial e é amplamente utilizada na indústria para controle de qualidade. No entanto, não representa fielmente a viscosidade real quando o fluido é não newtoniano ou quando há efeitos de extremidade e deslizamento significativos.

\vspace{1em}

\subsection*{7.2 Viscosidade Verdadeira}
\begin{equation}
\eta = \frac{\tau_w}{\dot{\gamma}_w} \quad \left[ \si{Pa.s} \right]
\end{equation}

\textbf{Onde:}
\begin{itemize}[leftmargin=1.5em]
  \item $\eta$ – Viscosidade verdadeira (\si{Pa.s})
  \item $\tau_w$ – Tensão de cisalhamento na parede (corrigida por Bagley, se aplicável) (\si{Pa})
  \item $\dot{\gamma}_w$ – Taxa de cisalhamento real na parede (corrigida por Mooney e W-R, se aplicáveis) (\si{s^{-1}})
\end{itemize}

\textbf{Finalidade e Interpretação:}

Esta equação fornece a viscosidade mais representativa do comportamento intrínseco do material. Ela é mais fiel à realidade, pois considera o verdadeiro perfil de velocidade no interior do capilar e remove artefatos experimentais. Essa viscosidade é usada para ajustar modelos reológicos e prever o comportamento do material em diferentes condições de escoamento.


\begin{table}[h!]
\centering
\caption{Comparação entre viscosidade aparente e verdadeira}
\begin{tabular}{@{}lll@{}}
\toprule
\textbf{Tipo} & \textbf{Base de Cálculo} & \textbf{Aplicação Principal} \\
\midrule
Aparente (\(\eta_a\)) & \(\tau_w\) (bruta), \(\dot{\gamma}_{aw}\) (bruta) & Controle rápido, estimativas iniciais \\
Verdadeira (\(\eta\)) & \(\tau_w\) (corrigida), \(\dot{\gamma}_w\) (corrigida) & Modelagem, predição, análise científica \\
\bottomrule
\end{tabular}
\end{table}



\section*{8. Modelos Reológicos Ajustados}

Após corrigir os dados experimentais para obter pares $(\tau_w, \dot{\gamma}_w)$ confiáveis, ajustam-se modelos matemáticos para descrever o comportamento reológico do material. Esses modelos permitem representar a relação entre tensão e taxa de cisalhamento de maneira funcional, possibilitando previsões e comparações entre diferentes materiais, bem como estimar parâmetros reológicos que caracterizem o tipo de fluido, como viscosidade, tensão de escoamento e pseudoplasticidade. O script ajusta todos os modelos abaixo e seleciona o melhor com base no coeficiente de determinação ($R^2$).

Os modelos reológicos foram ajustados por meio de regressão não linear utilizando o método de mínimos quadrados (via \texttt{curve\_fit}). A seleção do melhor modelo baseou-se no coeficiente de determinação $R^2$, obtido a partir da comparação entre os valores experimentais $\tau_w$ e os valores previstos pelos modelos. Apenas pontos com $\tau_w > 0$ e $\dot{\gamma}_w > 0$ foram considerados válidos para o ajuste.

O comportamento reológico foi classificado com base no valor do expoente \( n \) ajustado no modelo de Lei da Potência ou Herschel-Bulkley. Os critérios utilizados foram:
\begin{table}[h!]
\centering
\caption{Interpretação do índice de comportamento de escoamento \(n\)}
\begin{tabular}{@{}ll@{}}
\toprule
Valor de \(n\) & Comportamento do fluido \\ \midrule
\(n \approx 1\) & Fluido newtoniano \\
\(n < 1\)       & Fluido pseudoplástico (afinamento com o aumento da taxa de cisalhamento) \\
\(n > 1\)       & Fluido dilatante (espessamento com o aumento da taxa de cisalhamento) \\
\bottomrule
\end{tabular}
\end{table}


Essa classificação fornece uma visão qualitativa sobre a estrutura e dinâmica do fluido durante o escoamento.




\vspace{1em}

\subsection*{8.1 Modelo Newtoniano}
\begin{equation}
\tau_w = \eta \, \dot{\gamma}_w
\end{equation}

\textbf{Onde:}
\begin{itemize}[leftmargin=1.5em]
  \item $\tau_w$ – Tensão de cisalhamento (\si{Pa})
  \item $\dot{\gamma}_w$ – Taxa de cisalhamento real (\si{s^{-1}})
  \item $\eta$ – Viscosidade constante (\si{Pa.s})
\end{itemize}

\textbf{Descrição:} Modelo mais simples. Assume que a viscosidade é constante, independentemente da taxa de cisalhamento. Válido para fluidos ideais como água ou óleos leves. Raramente adequado para pastas cerâmicas.

\vspace{1em}

\subsection*{8.2 Lei da Potência (Power Law)}
\begin{equation}
\tau_w = K \, \dot{\gamma}_w^n
\end{equation}

\textbf{Onde:}
\begin{itemize}[leftmargin=1.5em]
  \item $K$ – Coeficiente de consistência (\si{Pa.s^n})
  \item $n$ – Índice de comportamento de fluxo (adimensional)
\end{itemize}

\textbf{Descrição:} Captura o comportamento de fluidos cuja viscosidade varia com a taxa de cisalhamento. Para $n<1$, o fluido é pseudoplástico (afinamento); para $n>1$, dilatante (espessamento). Não representa tensão de escoamento.

\vspace{1em}

\subsection*{8.3 Modelo de Bingham}
\begin{equation}
\tau_w = \tau_0 + \eta_p \, \dot{\gamma}_w
\end{equation}

\textbf{Onde:}
\begin{itemize}[leftmargin=1.5em]
  \item $\tau_0$ – Tensão de escoamento (\si{Pa})
  \item $\eta_p$ – Viscosidade plástica (\si{Pa.s})
\end{itemize}

\textbf{Descrição:} Indica que o material não começa a fluir até que uma tensão mínima ($\tau_0$) seja ultrapassada. Após isso, flui com viscosidade constante ($\eta_p$). Útil para materiais como argilas e suspensões concentradas.

\vspace{1em}

\subsection*{8.4 Modelo de Herschel-Bulkley}
\begin{equation}
\tau_w = \tau_0 + K \, \dot{\gamma}_w^n
\end{equation}

\textbf{Onde:}
\begin{itemize}[leftmargin=1.5em]
  \item $\tau_0$ – Tensão de escoamento (\si{Pa})
  \item $K$ – Coeficiente de consistência (\si{Pa.s^n})
  \item $n$ – Índice de comportamento de fluxo (adimensional)
\end{itemize}

\textbf{Descrição:} Generaliza os modelos anteriores. Apresenta tensão de escoamento inicial ($\tau_0$), seguida por comportamento não newtoniano dependente de $n$. É o modelo mais versátil e frequentemente o melhor ajustado para pastas cerâmicas e suspensões complexas.

\vspace{1em}

\subsection*{8.5 Modelo de Casson}
\begin{equation}
\sqrt{\tau_w} = \sqrt{\tau_0} + \sqrt{\eta_{cas}} \sqrt{\dot{\gamma}_w}
\end{equation}

\textbf{Onde:}
\begin{itemize}[leftmargin=1.5em]
  \item $\tau_0$ – Tensão de escoamento (\si{Pa})
  \item $\eta_{cas}$ – Viscosidade de Casson (\si{Pa.s})
\end{itemize}

\textbf{Descrição:} Modelo amplamente utilizado para descrever o comportamento de fluidos biológicos (como sangue) e alimentos (como chocolate derretido), mas que também se aplica bem a certas suspensões cerâmicas concentradas. Ele assume uma transição mais suave entre a região de escoamento e a de fluxo.

\vspace{1em}

O script calcula o coeficiente de determinação ($R^2$) para cada modelo e escolhe aquele que melhor representa os dados experimentais corrigidos.


\section*{9. Fluxo de Trabalho e Arquitetura do Sistema}

O sistema de análise reológica foi reestruturado em módulos independentes para garantir maior robustez, rastreabilidade e facilidade de manutenção. Cada script desempenha uma função específica no ciclo de vida dos dados, desde a coleta até a análise comparativa.

\href{https://www.mermaidchart.com/d/6f9e496c-626c-429a-ba93-17c761bf78d0}{\textbf{Clique aqui para visualizar o Fluxograma do Processo Atualizado}}

\subsection*{9.1 Coleta de Dados (Script 1: Controle\_Reometro.py)}
Responsável pela interface com o hardware (Arduino/HX711).
\begin{itemize}
  \item \textbf{Função:} Leitura em tempo real dos sensores de pressão (Linha e Pasta) e da balança.
  \item \textbf{Validação:} Implementa verificação automática de estabilidade e permite ao usuário rejeitar pontos espúrios durante o ensaio.
  \item \textbf{Saída:} Gera um arquivo JSON bruto contendo os dados experimentais e metadados do ensaio.
\end{itemize}

\subsection*{9.2 Pré-Processamento (Script 1a: Unir\_Arquivos\_Json.py)}
Módulo opcional para organização de dados.
\begin{itemize}
  \item \textbf{Função:} Permite fundir múltiplos arquivos JSON de ensaios parciais em um único arquivo consolidado.
  \item \textbf{Aplicação:} Útil quando um ensaio é interrompido e retomado posteriormente, garantindo que a análise reológica trate o conjunto como um todo.
\end{itemize}

\subsection*{9.3 Análise Reológica (Script 2: Analise\_reologica.py)}
O núcleo matemático do sistema. Processa os dados brutos e aplica as correções físicas.
\begin{itemize}
  \item \textbf{Entrada:} Arquivo JSON gerado pelo Script 1 ou 1a.
  \item \textbf{Processamento:}
  \begin{enumerate}
      \item Conversão de unidades para o SI;
      \item Cálculo de tensões e taxas aparentes;
      \item Aplicação das correções de Bagley e Mooney (se dados disponíveis);
      \item Correção de Weissenberg-Rabinowitsch;
      \item Ajuste dos modelos reológicos (Newton, Power Law, Bingham, Herschel-Bulkley, Casson).
  \end{enumerate}
  \item \textbf{Saída:} Arquivos CSV com os dados processados e JSON com os parâmetros dos modelos ajustados.
\end{itemize}

\subsection*{9.4 Tratamento Estatístico (Script 2b: Tratamento\_Estatistico.py)}
Módulo de confiabilidade (detalhado na Seção 10).
\begin{itemize}
  \item \textbf{Função:} Analisa a dispersão dos dados experimentais.
  \item \textbf{Método:} Agrupa dados por pressão nominal, remove outliers estatísticos ($> 2\sigma$) e calcula o Coeficiente de Variação (CV).
  \item \textbf{Saída:} Relatório de qualidade dos dados, indicando a precisão do ensaio.
\end{itemize}

\subsection*{9.5 Visualização e Comparação (Scripts 3 e 4)}
Ferramentas de pós-processamento para interpretação dos resultados.
\begin{itemize}
  \item \textbf{Script 3 (Visualizar\_resultados.py):} Gera gráficos detalhados de um único ensaio, incluindo curvas de fluxo, viscosidade e resíduos dos modelos.
  \item \textbf{Script 4 (Comparativo-Analises.py):} Permite sobrepor curvas de múltiplos ensaios para comparação direta e calcula o erro relativo (MAPE) entre uma amostra de referência e outras amostras (detalhado na Seção 11).
\end{itemize}
\section*{10. Tratamento Estatístico de Dados (Script 2b)}

Para garantir a confiabilidade dos dados experimentais, especialmente quando há múltiplas repetições para uma mesma pressão aplicada, o sistema implementa um módulo de tratamento estatístico robusto (Script 2b).

\subsection*{10.1 Agrupamento e Filtragem de Outliers}
Os dados brutos são primeiramente agrupados por níveis nominais de pressão aplicada ($P_{ext}$). Dentro de cada grupo, aplica-se um filtro estatístico para remover \textit{outliers} (pontos espúrios) que se desviam significativamente da média. O critério utilizado é baseado no desvio padrão ($\sigma$) da taxa de cisalhamento:

\begin{equation}
\text{Se } | \dot{\gamma}_{w,i} - \mu_{\dot{\gamma}} | > 2\sigma_{\dot{\gamma}} \implies \text{Ponto Descartado}
\end{equation}

Isso remove pontos afetados por bolhas de ar, entupimentos momentâneos ou erros de leitura, mantendo apenas os dados consistentes.

\subsection*{10.2 Cálculo de Médias e Incertezas}
Para cada nível de pressão, são calculadas as médias aritméticas ($\mu$) e os desvios padrão ($\sigma$) das grandezas principais ($\tau_w, \dot{\gamma}_w, \eta$).

\begin{equation}
\mu = \frac{1}{N} \sum_{i=1}^{N} x_i \quad ; \quad \sigma = \sqrt{\frac{\sum_{i=1}^{N} (x_i - \mu)^2}{N-1}}
\end{equation}

\subsection*{10.3 Coeficiente de Variação e Propagação de Erro}
A qualidade do ensaio é avaliada pelo Coeficiente de Variação (CV), que expressa a dispersão relativa dos dados:

\begin{equation}
CV_{\%} = \left( \frac{\sigma}{\mu} \right) \times 100
\end{equation}

Para a viscosidade ($\eta = \tau_w / \dot{\gamma}_w$), a incerteza é estimada pela propagação de erros dos CVs da tensão e da taxa de cisalhamento:

\begin{equation}
CV_{\eta} \approx \sqrt{ (CV_{\tau_w})^2 + (CV_{\dot{\gamma}_w})^2 }
\end{equation}

Um $CV_{\eta}$ baixo ($< 10\%$) indica alta reprodutibilidade e confiabilidade dos resultados reológicos.

\section*{11. Análise Comparativa e Discrepância (Script 4)}

O módulo comparativo permite sobrepor curvas de diferentes ensaios e quantificar as diferenças entre elas.

\subsection*{11.1 Metodologia de Comparação}
O script carrega os modelos ajustados de múltiplos ensaios e gera gráficos comparativos de $\tau_w$ vs $\dot{\gamma}_w$ e $\eta$ vs $\dot{\gamma}_w$. Isso permite visualizar rapidamente o efeito de diferentes formulações ou condições de processamento.

\subsection*{11.2 Análise de Discrepância (MAPE)}
Para quantificar a diferença entre uma amostra de teste e uma amostra de referência, utiliza-se o Erro Percentual Absoluto Médio (MAPE). O script gera uma série de pontos virtuais na faixa de cisalhamento comum entre as amostras e calcula:

\begin{equation}
MAPE (\%) = \frac{100}{N} \sum_{j=1}^{N} \left| \frac{Y_{teste,j} - Y_{ref,j}}{Y_{ref,j}} \right|
\end{equation}

Onde $Y$ pode ser a tensão de cisalhamento ($\tau_w$) ou a viscosidade ($\eta$).
\begin{itemize}
    \item \textbf{MAPE $<$ 5\%}: As amostras são consideradas reologicamente equivalentes.
    \item \textbf{MAPE $>$ 10\%}: Indica diferenças significativas no comportamento do fluxo.
\end{itemize}


    


Esses gráficos estão salvos automaticamente na pasta de resultados com nomes descritivos (timestamp data/hora) e podem ser utilizados para visualização complementar dos ajustes.






\end{document}
